\documentclass{ferseminar}



\student{Antonio Benc, Matija Herceg, Luka Skukan}
\voditelj{doc. dr. sc. Mirjana Domazet-Lošo}
\mjestodatum{Zagreb}{siječanj}{2016}
\naslov{Algoritam za ažuriranje Burrows-Wheelerove transformacije u četiri koraka}


\begin{document}
\stvoripredstranice
\section{Uvod}
Burrows-Wheelerova transformacija (BWT) je transformacija teksta, vrlo prikladna za kompresiju. Korištena je u nekim popularnim alatima za kompresiju bez gubitaka, primjerice programu bzip2.  Osim pod nazivom Burrows-Wheelerova transformacija, poznata je i pod nazivom \textit{block-sorting compression}. 
\subparagraph{}
Konceptualno, tekst nad kojim je izvršena BWT je sličan sufiksnom polju. Zbog te sličnosti BWT se koristi i kao indeksna struktura. BWT teksta $T$ ($bwt(t)$) se često dobiva iz modifikacije sufiksnog polja koja konstrukcija ima $O(n)$ složenost. Pohranjivanje sufiksnog polja u memoriji je jos uvjek glavni problem jer zahtjeva n $n\log{}n$ bita dok pohrana BWT-a u memoriju zahtjeva $(n\log{}\sigma)$ bita, gdje je $\sigma$ broj slova u abecedi.
\subparagraph{}
U ovom seminaru razmatrat će se uobičajne operacije nad tekstom, umetanje znakova, brisanje znakova ili mijenjanje znaka, koje tekst $T$ transformiraju u novi tekst $T'$. Biti će proučavan utjecaj tih operacija na $bwt(T)$ i biti će predložen algoritam za pretvorbu $bwt(T)$ u $bwt(T')$.

\section{Poglavlja seminara}

\section{Zaključak}
\dodajliteraturu{bazaLiterature}
\section{Sažetak}
\end{document}

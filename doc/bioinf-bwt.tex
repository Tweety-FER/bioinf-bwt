\documentclass{ferseminar}

\usepackage[table,xcdraw]{xcolor}


\student{Antonio Benc, Matija Herceg, Luka Skukan}
\voditelj{doc. dr. sc. Mirjana Domazet-Lošo}
\mjestodatum{Zagreb}{siječanj}{2016}
\naslov{Algoritam za ažuriranje Burrows-Wheelerove transformacije u četiri koraka}


\begin{document}
\stvoripredstranice
\section{Uvod}
Burrows-Wheelerova transformacija (BWT) je transformacija teksta, vrlo prikladna za kompresiju. Korištena je u nekim popularnim alatima za kompresiju bez gubitaka, primjerice programu bzip2.  Osim pod nazivom Burrows-Wheelerova transformacija, poznata je i pod nazivom \textit{block-sorting compression}. 
\subparagraph{}
Konceptualno, tekst nad kojim je izvršena BWT je sličan sufiksnom polju. Zbog te sličnosti BWT se koristi i kao indeksna struktura. BWT teksta $T$ ($bwt(t)$) se često dobiva iz modifikacije sufiksnog polja koja konstrukcija ima $O(n)$ složenost. Pohranjivanje sufiksnog polja u memoriji je jos uvjek glavni problem jer zahtjeva n $n\log{}n$ bita dok pohrana BWT-a u memoriju zahtjeva $(n\log{}\sigma)$ bita, gdje je $\sigma$ broj slova u abecedi.
\subparagraph{}
U ovom seminaru razmatrat će se uobičajne operacije nad tekstom, umetanje znakova, brisanje znakova ili mijenjanje znaka, koje tekst $T$ transformiraju u novi tekst $T'$. Biti će proučavan utjecaj tih operacija na $bwt(T)$ i biti će predložen algoritam za pretvorbu $bwt(T)$ u $bwt(T')$.
\section{Burrows-Wheelerova transformacija}
Neka je tekst $T=T[0..n]$ riječ duljine $n+1$ s abecedom $\Sigma$, pri čemu je abeceda $\Sigma$ konacne velicine $\sigma$. Zadnji znak u rijeci $T$ je jedinstveni znak $\$$ (\textit{sentinel}) koji ima vrijedmost manju od svih ostalih znakova u abecedi.  Podniz koji pocinje na poziciji $i$ i zavrsava na poziciji $j$ je oznacen s $T[i..j]$, znak na poziciji $i$ je oznacen s $T[i]$ te je ciklički pomak reda $i$,  $T[i..n]T[0..i-1]$, oznacen s $T^{[i]}$.
\subparagraph{}
Burrows-Wheelerova transformacija od $T$, oznacena s $bwt(T)$, je tekst duljine n+1 koji odgovara zadnjem stupcu, ($L$), matrice čiji su reci leksikografski sortirani ciklički pomaci $T^{[i]}$. Prvi stupac matrice, ($F$), je sortiran, tako da se jednostavno može izračunati iz stupca $L$. Redovi sortiranih cikličkih pomaka, $\pi$ jednaki su sufiksnom polju od $T$. Iz toga slijedi kako su sufiksno polje $SA[i]$ i $L$ povezani jednostavnom formulom $L[i]= T[(SA[i]-1) \mod |T|]$.
\subparagraph{}
\begin{table}[h]

\begin{tabular}{r c c c c c c c c c r c c c c c c c c c c}
		
		  &   &  &  &  &  &  	
      & & & & & & & & 
      & & & \multicolumn{1}{l|}{i} & $\pi$    \\ 
	  \cline{2-7} \cline{11-16} \cline{18-19}
      \multicolumn{1}{l|}{$T^{[0]}$} & A & T & G & C & G & \multicolumn{1}{l|}{\$} &  	
      & &    \multicolumn{1}{l|}{$T^{[5]}$} & \$ & A & T & G & C & \cellcolor[HTML]{9B9B9B} G
      & & \multicolumn{1}{l|}{0} & 5    \\ 
      
	 \multicolumn{1}{l|}{$T^{[1]}$} & T & G & C & G & \$ & \multicolumn{1}{l|}{A} &
	  & &    \multicolumn{1}{l|}{$T^{[0]}$} & A & T & G & C & G & \cellcolor[HTML]{9B9B9B} \$
	  & & \multicolumn{1}{l|}{1} & 0     \\ 
	  
	  \multicolumn{1}{l|}{$T^{[2]}$} & G & C & G & \$ & A & \multicolumn{1}{l|}{T} & 
	  & &    \multicolumn{1}{l|}{$T^{[3]}$} & C & G & \$ & A & T & \cellcolor[HTML]{9B9B9B} G
	  & & \multicolumn{1}{l|}{2} & 3     \\ 
	  
	  \multicolumn{1}{l|}{$T^{[3]}$} & C & G & \$ & A & T & \multicolumn{1}{l|}{G} & $\Rightarrow$
	  & &    \multicolumn{1}{l|}{$T^{[4]}$} & G & \$ & A & T & G & \cellcolor[HTML]{9B9B9B} C
	  & $\Rightarrow$ & \multicolumn{1}{l|}{3} & 4     \\ 
	  
	  \multicolumn{1}{l|}{$T^{[4]}$} & G & \$ & A & T & G & \multicolumn{1}{l|}{C} & 
	  & &   \multicolumn{1}{l|}{$T^{[2]}$} & G & C & G & \$ & A & \cellcolor[HTML]{9B9B9B} T
	  & & \multicolumn{1}{l|}{4} & 2     \\
	  
	  \multicolumn{1}{l|}{$T^{[5]}$} & \$ & A & T & G & C & \multicolumn{1}{l|}{G} &	
	  & &    \multicolumn{1}{l|}{$T^{[1]}$} & T & G & C & G & \$ & \cellcolor[HTML]{9B9B9B} A
	  & & \multicolumn{1}{l|}{5} & 1     \\
	  
	   \cline{2-7} \cline{11-16}
	  			&	 &	 &	 &	 & 	 &					& & &			  &$F$&	  &	  &	  & & $L$	  \\
 
\end{tabular}
\caption{Burrows-Wheelerova transformacija niza $ATGCG\$$}
\label{tablica:bwt}	
\end{table}


\section{Zaključak}
\dodajliteraturu{bazaLiterature}
\section{Sažetak}
\end{document}
